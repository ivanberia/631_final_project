\documentclass[10pt]{article}
\usepackage[pdftex]{graphicx}
\usepackage[utf8]{inputenc}
\usepackage{amsmath}
\usepackage{enumerate}
\usepackage{alltt}
\usepackage{float}
\usepackage{bbold}
\usepackage{caption}
\usepackage{subcaption}
\usepackage{subfig}
\restylefloat{table}
\usepackage{appendix}

\title{Symbolic Execution in Ruby\\
CMSC 631 Final Project}
\author{Elizabeth McNany and David Wasser}
\date{May 16, 2013}

\begin{document}
\maketitle

\begin{abstract}
We present a dynamic symbolic executor for the scripting language Ruby.  We use a proxy class to wrap variables during execution, and the Z3 SMT solver to find and solve constraints on symbolic variables.  The system was designed to be easy to add to an existing program and has a simple interface for variable creation and assertions.
\end{abstract}

\section{Introduction}
Security has become a major concern in application development in recent years.  Insecure software has been exploited and leveraged to steal large amount of money or information from various private and public organizations.  It has also been used to stop the operation of various websites or companies, resulting in large losses in revenue and damage to public image.\\

We are proposing an implementation of dynamic symbolic execution to help mitigate these vulnerabilities within Ruby applications.  This approach would allow developers to confirm that values are guaranteed to be within a specified acceptable range or that certain values are unaffected by other input values during the execution of the program.\\

Symbolic execution is a method of finding all possible outputs and paths of a program.  This is done by treating each variable as a symbolic variable that is a function of other variables in the program instead of a single value.  This allows a developer to confirm that a value will always be within a certain acceptable range, that a value will never influence another variable, or that a program will never execute certain code or run in an unintended way.  Symbolic execution will follow every single possible branch of a program and run it to completion or failure.  Clearly, this approach can have a number of large problems resulting from complexity in the program and the exponential growth of execution paths at each branch.\\

Symbolic execution for security verification in Ruby software is nothing new \cite{rails}.  While it has been done before, we could not find one that followed our exact approach of implementing symbolic execution in Ruby to test for potential errors or assertion failures.\\

\section{Methodology}
Our approach is to create a symbolic variable for a couple specified variables and then store this symbolic form along with the real value.  We will not follow every single path that is possible in the program, but will instead let it run as normal.  We will keep track of all operations done on the symbolic variables throughout the execution.  The user will also be able to add assertions, and as the program runs, our software will check all possible values of the symoblic variables against the assertions to verify that they will be satisfied.  While this approach is not as rigorous as true symbolic execution, it has many advantages.  This approach will run much faster and will be able to run to completion whenever the program can, without the performance problems caused to normal symbolic executors by loop structures along the way.  This apporach is also much faster and easier to implement.  It also can be implemented within Ruby instead of having to create a new interpreter or very large library to be able to implement this with Ruby.\\

While not identical, our approach to implement this in Ruby was similar to that of \cite{typeinf} in that we used a proxy system act a an intermediary between the program and the variables.  This allowed us to store the true variables as well as the symbolic variables and make similar operations on each as the code executed.\\


\subsection{Proxies}
We use proxies as a wrapper around objects to bundle the actual value and symbolic variable during execution.  Typically, a variable name points to its value in working memory as shown in Figure \ref{pointer:1}.  We add an additional layer with the proxy wrapper, as in Figure \ref{pointer:2}.  The program accesses only the proxy directly, which contains a reference to the variable, which then points to the actual value.

\begin{figure}
  \centering
  \begin{subfigure}{0.5\textwidth}
	\centering
	\includegraphics[height=25px]{pointer1.png}
	\caption{Variable in original program.}
	\label{pointer:1}
  \end{subfigure}\begin{subfigure}{0.5\textwidth}
	\centering
	\includegraphics[height=25px]{pointer2.png}
	\caption{Proxied variable representation.}
	\label{pointer:2}
  \end{subfigure}
  \caption{Representation of a normal vs. proxied variable.}
\end{figure}

This is accomplished via Ruby's built-in \texttt{coerce} and \texttt{method\_missing} classes.  These methods are part of the Ruby language and implementation and are called on a variable if it does not have particular properties.  By overriding these functions, we can catch method calls using the proxied variable and store the variable information, argument, and function at the time of call to pass on to the SMT solver.\\

Specifically, \texttt{method\_missing} is called on the object if the object does not have a particular method defined, passing in the information for the original function call.  The default behavior of the Ruby interpreter is to simply raise an exception.\\

When an object that has been passed as an argument to another method which requires a specific type, but the object does not have a defined conversion to that type, the \texttt{coerce} method is called on that object.  If no suitable conversion is found, an exception will be raised.\\

\subsection{Z3}
Z3 is a SMT solver developed and released by Microsoft.  We use the Z3 programming API check satisfiability of the environment given a set of symbolic variables and contstraints.  This allows us to check whether or not it is possible for an assertion to fail at a certain point within a program.\\

\section{Implementation}
There are two main components of our implementation: the Z3 API and the proxy class for variables.  The Z3 API is the interface to the Z3 SMT solver, which is used by the proxy class to create, manipulate, and query about symbolic variables.\\

\subsection{Z3-Ruby API}
APIs for Z3 are publicly available for C, C++, .NET, and Python \textemdash but not Ruby.  Instead, we wrote an interface for the Z3 C bindings in Ruby, using the Ruby FFI library.  The FFI library is used to create bindings for existing C function in Ruby without needed to re-compile any of the C code.  This requires the use of \texttt{MemoryPointer}s to be passed as pointers into the Ruby bindings for the functions where pointers would be required.  We also created some other helper functions to be able to interface more easily with the Z3 library for simple tasks such as creating integers, arrays, or booleans.  Within Z3, we used their integer implementation to model Ruby \texttt{Fixnum}s instead of bitvectors, which is the more common approach to model integers in symbolic execution.  This is because integers within Ruby are more similar to true integers than bitvectors.  While a \texttt{Fixnum} is a fixed size like a standard integer, if it grows too large, instead of overflowing and becoming negative or small again, it will instead become a \texttt{Bignum} and be able to continue to grow.\\

\subsection{Proxy Classes}
We created a general wrapper for proxied variables and two derived classes specific to \texttt{Fixnum}s and \texttt{Boolean}s.  The proxy class allows the programmer to initialize proxied variables with a particular value.  Methods of the original variable can be called and return values normally, and proxied variables passed as arguments to functions will be coerced into their original type.  However, if the proxied variable is a supported class, when calling a method it will first make a call to Z3 with a symbolic variable and the appropriate method and arguments.  Any arguments which are also proxied variables will be replaced with the corresponding symbolic variable; otherwise, arguments will be treated as a literal.  The burden is hence on the programmer to ensure that all pertinent variables have been enclosed in a proxy wrapper.\\

\section{Assessment}
We verified our implementation on several different small test programs.\\

One goal of our implementation was simplicity of use.  The example provided in Appendix \ref{appdx1} is typical and shows a direct comparison between the original program and the modifications required to use proxied variables.  The only modification to the actual variables is changing the integer inputs to be wrapped in \texttt{FixnumProxy}s.  The remainder of the program is left as-is and runs normally.  However, we can now insert assertions using the symbolic variables, which will make calls to the Z3 solver and return the result based on the variable's constraints.

\section{Conclusion}
We presented a dynamic symbolic execution system  that we have implemented in Ruby.  We used Z3 and a proxy system to keep track of symbolic variables throughout the execution of the program and currently can keep track of integers and booleans within this environment.  Future work on this subject would be to add more supported types in Ruby to our implementation.  This would simply require adding mappings from the original methods for the types to the Z3 fucntions that complete the same action for simple types that exist within Z3 as well.  For more complicated types, our plan is to treat a complex object as a set of simple types and then store symbolic copies of these simpler types.\\

\begin{thebibliography}{99}
\bibitem{rails}
A. Chaudhuri and J. Foster, ``Symbolic Security Analysis of Ruby-on-Rails Web Applications,'' in Proceedings of the 17th ACM Conference on Computer and Comm. Security, 2010, pp. 585-594.

\bibitem{typeinf}
B. Ren, J. Toman, T. S. Strickland, J. Foster, ``The Ruby Type Checker,'' in SAC '13: Proceedings of the 28th Annual ACM Symposium on Applied Computing, 2013.

\end{thebibliography}

\newpage
\appendix
\section{Program 1}
\label{appdx1}
Below is a simple program to perform long division on two integer inputs.  The first method, \texttt{main}, is the original version of the program.  The second method, \texttt{mainProxied}, is the same method, but modified to use proxied variables instead and check the program's output.\\

\begin{verbatim}
# takes two inputs (dividend, divisor) and outputs the quotient and remainder
# version of http://rubyquiz.strd6.com/quizzes/180-long-division

load "proxyclass.rb"

def main(dividend, divisor)
	quotient = 0
	remainder = dividend
	Math.log10(dividend).ceil.downto(0) do |exp|
		magnitude = 10 ** exp
		trydiv, rest = dividend.divmod(magnitude)
		if trydiv >= divisor
			quotient_digit, remainder = trydiv.divmod(divisor)
			quotient += quotient_digit * magnitude
			dividend = (remainder * magnitude + rest)
			break
		end
	end
	return quotient, remainder
end

def mainProxied(dd, dv)
	dividend = FixnumProxy.new(dd)
	divisor = FixnumProxy.new(dv)
	quotient = FixnumProxy.new(0)
	remainder = FixnumProxy.new(dd)
	Math.log10(dividend.to_f).ceil.downto(0) do |exp|
		magnitude = 10 ** exp
		trydiv, rest = dividend.divmod(magnitude)
		if trydiv >= divisor
			quotient_digit, remainder = trydiv.divmod(divisor)
			quotient += quotient_digit * magnitude
			dividend = (remainder * magnitude + rest)
			break
		end
	end
	
	# the remainder should always be less than the divisor
	puts ProxyClass.assert_less_than(remainder, divisor)
	
	# similarly, this should be false
	puts ProxyClass.assert_greater_than_or_equal(remainder, divisor)
	
	# divisor should not be zero
	puts ProxyClass.assert_not_equal(divisor, 0)
	
	return quotient, remainder
end

# get command-line arguments
dd = ARGV[0].to_i
dv = ARGV[1].to_i
if dv != 0
	q, r = main(dd, dv)
	puts "Quotient: #{q}, Remainder: #{r}"
	
	q, r = mainProxied(dd, dv)
	puts "Proxied Quotient: #{q}, Remainder: #{r}"
else
	puts "Error: div by zero"
end
\end{verbatim}

\end{document}
